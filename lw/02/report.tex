\documentclass{bsuir}
\usepackage{makecell}
\usepackage{multirow}

\departmentlong{инженерной психологии и эргономики}
\worktitle{Лабораторная работа \textnumero2\\\textquote{Разработка структуры взаимодействия \textquote{пользователь-интерфейс}}}
\titleleft{
    Проверила:\\
    Яцкевич А. Ю.\\
    ~
}
\titleright{
    Выполнил:\\
    Бородин А.Н.\\
    гр. 310901
}
\titlepageyear{2025}

\usepackage{pgfplots}
\usepackage{amsmath}
\usepackage{breqn}

\newlength{\tablewidth}
\setlength{\tablewidth}{\textwidth - \parindent}

\renewcommand*{\thesection}{\arabic{section}}

\begin{document}

\maketitle
\mainmatter

Проектируемый продукт "--- IDE, совмещенная с обучающей платформой для студентов
БГУИР.

\section{Пользовательские сценарии}

\subsection*{Пользовательская история (User Story)}

\textbf{Персонаж:} Витька Лаптев, студент-новичок.

\textbf{Формулировка:}
\textit{«Я, как студент первого курса, хочу быстро начать работать над лабораторной по программированию на C, чтобы не тратить время на установку компиляторов. Вчера я потратил два часа, пытаясь подключить MinGW, но всё сломалось. Хотелось бы, чтобы при первом запуске редактор сам устанавливал всё необходимое и показывал все задания.»}

\subsection*{Концептуальный сценарий (Conceptual Scenario)}

\textbf{Цель:} Интеграция учебных материалов и автоматизация проверки заданий.

\textbf{Описание:}

\begin{itemize}
    \item Система объединяет среду разработки с платформой СЭО БГУИР.
    \item Студенты получают задания напрямую в редакторе.
    \item Код выполняется в предварительно настроенных контейнерах.
    \item Преподаватели создают тесты для автоматической проверки.
    \item Результаты отображаются в дашбордах с рекомендациями.
\end{itemize}

\subsection*{Конкретный сценарий (Concrete Scenario)}

\textbf{Персонаж:} Ольга Сергеевна, преподаватель"=программист.

\textbf{Действия:}
\begin{enumerate}
    \item Создает задание по C\# в разделе «Курсы».
    \item Загружает шаблон проекта с unit-тестами.
    \item Студенты получают уведомление в редакторе.
    \item Система запускает тесты в контейнере после отправки кода.
    \item Результаты проверки: 25 работ успешны, 3 "--- с ошибками.
    \item Автоматическая отправка комментариев студентам.
\end{enumerate}

\subsection*{Вариант использования (Use Case)}

В таблице \ref{tab:use-cases} приведен сценарий отправки лабораторной работы
студентом.

{\small\begin{longtable}{|>{\centering\arraybackslash}m{0.2\linewidth}|>{\raggedright\arraybackslash}m{0.75\linewidth}|}
        \caption{Отправка лабораторной работы студентом}\label{tab:use-cases}                                                           \\
        \hline
        \textbf{Параметр}                             & \textbf{Значение}                                                               \\
        \hline
        \endfirsthead

        \caption*{Продолжение таблицы \ref{tab:use-cases}}                                                                              \\
        \hline
        \textbf{Параметр}                             & \textbf{Значение}                                                               \\
        \hline
        \endhead

        \hline
        \endfoot

        \hline
        \endlastfoot

        Участники                                     & Студент, Система                                                                \\\hline
        Предусловие                                   & Студент авторизован в IDE, задание доступно в разделе \textquote{Лабораторные}. \\\hline
        \multirow{5}{=}{\centering Основной сценарий} &
        1. Открывает задание \textquote{Лабораторная №2 по C}.                                                                          \\\cline{2-2}
                                                      & 2. Получает шаблон кода и инструкции.                                           \\\cline{2-2}
                                                      & 3. Пишет код с помощью IntelliSense.                                            \\\cline{2-2}
                                                      & 4. Нажимает «Отправить на проверку».                                            \\\cline{2-2}
                                                      & 5. Система запускает тесты и возвращает результат.                              \\\hline
        Альтернативный сценарий                       &
        \textbf{Шаг 5a:} Если тесты не пройдены, система выделяет ошибки и предлагает примеры исправлений.                              \\\hline
    \end{longtable}}

\subsection*{Agile User Story}

\textbf{Формат:}
\textit{«Как студент-практик, я хочу синхронизировать учебные проекты с GitHub, чтобы автоматически сохранять изменения и делиться кодом с командой.»}

\textbf{Критерии приемки:}

\begin{enumerate}
    \item Возможность привязки проекта к репозиторию GitHub.
    \item Автоматические коммиты при сохранении кода.
    \item Уведомления о конфликтах при пулле.
\end{enumerate}

\textbf{Технические заметки:}

\begin{itemize}
    \item Интеграция через GitHub API.
    \item Поддержка двухфакторной аутентификации.
\end{itemize}

\section{Выводы}

Разработка пользовательских сценариев позволила систематизировать ключевые
проблемы аудитории. Концептуальные сценарии устранили фрагментацию учебного
процесса, конкретные "--- детализировали технические аспекты, Agile User Stories
связали функционал с ценностью. Комплексный подход минимизировал риски
дисбаланса ресурсов.

\end{document}