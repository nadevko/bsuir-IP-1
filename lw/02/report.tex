\documentclass{bsuir}
\usepackage{makecell}
\usepackage{multirow}

\departmentlong{инженерной психологии и эргономики}
\worktitle{Лабораторная работа \textnumero2\\\textquote{Разработка структуры взаимодействия \textquote{пользователь-интерфейс}}}
\titleleft{
    Проверила:\\
    Яцкевич А. Ю.\\
    ~
}
\titleright{
    Выполнил:\\
    Бородин А.Н.\\
    гр. 310901
}
\titlepageyear{2025}

\usepackage{pgfplots}
\usepackage{amsmath}
\usepackage{breqn}

\newlength{\tablewidth}
\setlength{\tablewidth}{\textwidth - \parindent}

\renewcommand*{\thesection}{\arabic{section}}

\begin{document}

\maketitle
\mainmatter

Проектируемый продукт "--- IDE, совмещенная с обучающей платформой для студентов
БГУИР.

\section{Пользовательские сценарии}

\subsection*{Agile User Story}

\textbf{История:}
\textit{\textquote{Как студент-практик, я хочу синхронизировать учебные проекты
с GitHub, чтобы автоматически делиться изменениями кода с бригадой.}}

\textbf{Критерии приемки:}

\begin{enumerate}
    \item Возможность привязки проекта к репозиторию GitHub.
    \item Автоматические коммиты при сохранении кода.
    \item Уведомления о конфликтах при пулле.
\end{enumerate}

\textbf{Технические заметки:}

\begin{itemize}
    \item Интеграция через GitHub API.
    \item Поддержка двухфакторной аутентификации на GitHub.
\end{itemize}

\subsection*{Концептуальный сценарий (Conceptual Scenario)}

\textbf{Цель:} Упрощение настройки окружения.

\textbf{Описание:} \textit{\textquote{Студенты"=новички, часто сталкиваются с
трудностями при выполнении лабораторных работ, например, настройкой окружения
или подключением инструментов, таких как MinGW. Для решения этих проблем система
настраивает окружение разработчика, контейнеры и собирает код в готовом
окружении. Таким образом, у студента не должно возникать проблем с запуском
любого кода, на изучаемых в университете языках программирования.}}

\subsection*{Конкретный сценарий (Concrete Scenario)}

\textbf{История:} \textit{\textquote{Когда Ольга Сергеевна,
преподаватель"=программист, работает на личном, а не университетском компьютере,
она хочет создать задание по C\# в разделе \textquote{Курсы}. После этого она
загружает шаблон проекта с unit"=тестами, чтобы студенты могли приступить к
выполнению. Студенты должны получить уведомление в редакторе об изменениях, и
после отправки кода система автоматически запускает тесты в контейнере. После
этого Ольга Сергеевна должна увидеть результаты проверки и ответить на них,
приняв результат или отвергнув, опционально оценив и прокомментировав его.}}

\subsection*{Вариант использования (Use Case)}

В таблице \ref{tab:use-cases} приведен сценарий отправки лабораторной работы
студентом.

{\small\begin{longtable}{|>{\centering\arraybackslash}m{0.2\linewidth}|>{\raggedright\arraybackslash}m{0.75\linewidth}|}
    \caption{Отправка лабораторной работы студентом}\label{tab:use-cases}                                                                                                 \\
    \hline
    \textbf{Параметр}                             & \textbf{Значение}                                                                                                     \\
    \hline
    \endfirsthead

    \caption*{Продолжение таблицы \ref{tab:use-cases}}                                                                                                                    \\
    \hline
    \textbf{Параметр}                             & \textbf{Значение}                                                                                                     \\
    \hline
    \endhead

    \hline
    \endfoot

    \hline
    \endlastfoot

    Участники                                     & Студент                                                                                                               \\\hline
    Постусловие                                   & Система выводит результат проверки работы: успешно прошла и нужно только подтверждение преподавателя или есть ошибки. \\\hline
    \multirow{5}{=}{\centering Основной сценарий} &
    1. Открывает задание \textquote{Лабораторная №2 по C}.                                                                                                                \\\cline{2-2}
                                                  & 2. Получает шаблон кода и инструкции.                                                                                 \\\cline{2-2}
                                                  & 3. Пишет код с помощью IntelliSense.                                                                                  \\\cline{2-2}
                                                  & 4. Нажимает «Отправить на проверку».                                                                                  \\\cline{2-2}
                                                  & 5. Система запускает тесты.                                                                                           \\\hline
    Альтернативный сценарий                       & 5a. Если тесты не пройдены, система выделяет ошибки                                                                   \\\hline
\end{longtable}}

\section{Выводы}

Разработка пользовательских сценариев позволила систематизировать ключевые
проблемы аудитории. Концептуальные сценарии устранили фрагментацию учебного
процесса, конкретные "--- детализировали технические аспекты, Agile User Stories
связали функционал с ценностью. Комплексный подход минимизировал риски
дисбаланса ресурсов.

\end{document}