\documentclass{bsuir}
\usepackage{makecell}
\usepackage{multirow}

\departmentlong{инженерной психологии и эргономики}
\worktitle{Лабораторная работа \textnumero1\\\textquote{Разработка концепции продукта}}
\titleleft{
    Проверила:\\
    Яцкевич А. Ю.\\
    ~
}
\titleright{
    Выполнил:\\
    Бородин А.Н.\\
    гр. 310901
}
\titlepageyear{2025}

\usepackage{pgfplots}
\usepackage{amsmath}
\usepackage{breqn}

\newlength{\tablewidth}
\setlength{\tablewidth}{\textwidth - \parindent}

\renewcommand*{\thesection}{\arabic{section}}

\begin{document}

\maketitle
\mainmatter

Проектируемый продукт "--- IDE, совмещенная с обучающей платформой для студентов
БГУИР.

\section{Сфера применения проектируемого программного продукта}

Программный продукт относится к сфере образовательных технологий (EdTech) и
инструментов для разработки программного обеспечения. Он сочетает в себе
интегрированную среду разработки (IDE) и обучающую платформу, ориентированную на
студентов БГУИР. Сфера применения продукта охватывает учебный процесс по
дисциплинам, связанным с программированием, веб"=разработкой, встраиваемыми
системами и микроконтроллерами. Решение предназначено для использования как в
аудиторной работе, так и для самостоятельного обучения, обеспечивая поддержку
как студентов, так и преподавателей в рамках единой экосистемы.

Студенты БГУИР сталкиваются с необходимостью постоянно переключаться между
множеством инструментов: отдельными IDE (Vi"-sual Stu"-dio, Ec"-li"-pse,
PyCharm), платформами теории и выполнения заданий (СЭО), прочими
предмето"=специфичными ресурсами. Это приводит к фрагментации учебного процесса,
усложняет доступ к материалам и увеличивает время на освоение практических
навыков. Кроме того, выполнение лабораторных работ, к примеру по
embedded"=разработке, требует настройки специфических сред, что создает
технические сложности. Преподаватели, в свою очередь, испытывают трудности с
оперативной проверкой заданий, анализом прогресса студентов и интеграцией
учебных материалов в единое пространство, что снижает эффективность обратной
связи.

\section{Описание}

\subsection{Описание системы и её функционала}

В качестве основы для проектируемого решения выбрана среда разработки Visual
Studio Code (VS Code) "--- кроссплатформенный редактор кода с открытым исходным
кодом, поддерживающий множество языков программирования и технологий. VS Code
предоставляет базовые услуги для разработчиков: редактирование кода с подсветкой
синтаксиса, интеграцию с системами контроля версий (Git), отладку, расширяемость
через плагины, а также возможность кастомизации интерфейса. Благодаря поддержке
расширений, VS Code адаптируется под различные задачи "--- от веб"=разработки до
работы с микроконтроллерами. Однако его изначальная цель "--- общее
программирование, а не специализированная поддержка учебного процесса, что
создаёт ограничения для образовательных сценариев.

\subsection{Проблемы текущей системы}

Основная проблема VS Code в контексте обучения "--- отсутствие глубокой
интеграции с образовательными ресурсами и процессами. Студенты и преподаватели
вынуждены вручную переключаться между редактором кода, платформой Moodle (СЭО
БГУИР), браузером для поиска документации и веб"=разработки и отдельными
эмуляторами для embedded"=разработки. Это приводит к потере времени,
расфокусировке и усложнению выполнения лабораторных работ. Например, для
проверки заданий преподавателям приходится использовать внешние системы, что
замедляет обратную связь. Ещё одна проблема "--- отсутствие встроенной аналитики
прогресса: преподаватели не могут оперативно отслеживать успеваемость студентов,
а студенты "--- видеть персонализированные рекомендации по улучшению кода. Эти
недостатки были выявлены через опросы студентов БГУИР, анализ их рабочих
процессов и фидбэк от преподавателей, которые отмечали низкую эффективность
существующих инструментов в рамках учебных курсов.

\subsection{Предлагаемые решения и их обоснование}

Для устранения проблем предлагается модифицировать VS Code, интегрировав в него
функционал обучающей платформы:

\begin{itemize}
      \item
            Плагин для VSCode синхронизирующийся с СЭО БГУИР: он позволит
            студентам получать задания, лекции и дедлайны прямо в интерфейсе
            редактора, а преподавателям "--- создавать курсы и автоматически
            проверять работы через систему unit"=тестов.
      \item
            Процесс установки редактора будет модифицирован, чтобы предоставлять
            с первого старта уже настроенный редактор со всеми необходимыми
            расширениями, а также устанавливать вместе с ним все внешние
            зависимости расширений.
      \item
            Будет реализован упрощенный интерфейс работы с контейнерами для
            избежания добавления лишних внешних зависимостей, способствования
            кроссплатформенности и предоставление для всех студентов и
            преподавателей единого окружения, в котором программа и её тесты
            будут выполняться.
      \item
            Дополнительно может быть реализован плагин для создания отчётов и
            презентаций в пределах редактора, расчитанные не на широкий
            функционал, а на соответствие результата стандартам документации.
      \item
            У многих студентов есть проблемы с использованием системы контроля
            версий git и git"=хостингов (GitHub). Следовательно, есть смысл
            частично автоматизировать процесс и добавить систему подсказок.
\end{itemize}

\section{Анализ}

\subsection{Сравнение с аналогами}

Сравнение разрабатываемого продукта с аналогами приведено в таблице
\ref{tab:comparison}.

{\small\begin{longtable}{|>{\centering\arraybackslash}m{0.15\linewidth}|*{5}{>{\raggedright\arraybackslash}m{0.18\linewidth}|}}
      \caption{Сравнение разрабатываемого приложения с аналогами}\label{tab:comparison} \\
      \hline
      \textbf{Критерий}                                      &
      \textbf{Этот продукт}                                  &
      \textbf{VS Code}                                       &
      \textbf{Codio, Replit}                                 &
      \textbf{Moodle (СЭО)}                                                             \\
      \hline
      \endfirsthead

      \caption*{Продолжение таблицы \ref{tab:comparison}}                               \\
      \hline
      \textbf{Критерий}                                      &
      \textbf{Этот продукт}                                  &
      \textbf{VS Code}                                       &
      \textbf{Codio, Replit}                                 &
      \textbf{Moodle (СЭО)}                                                             \\
      \hline
      \endhead

      \hline
      \endfoot

      \hline
      \endlastfoot

      Интеграция с СЭО БГУИР                                 &
      Полная синхронизация заданий и материалов в IDE        &
      Отсутствует                                            &
      Отсутствует                                            &
      Ручная загрузка заданий в СЭО                                                     \\
      \hline

      Поддержка embedded"-разработки                         &
      Встроенные эмуляторы и шаблоны микроконтроллеров       &
      Возможна через плагины с настройкой                    &
      Ограниченная (фокус на веб-разработку)                 &
      Не поддерживается                                                                 \\
      \hline

      Автомати"-ческая проверка кода                         &
      Встроенные тесты в контейнерах                         &
      Зависит от плагинов (например, Code Runner)            &
      Есть, но без кастомизации                              &
      Ручная оценка через вложения                                                      \\
      \hline

      Аналитика прогресса                                    &
      Дашборды и рекомендации для студентов и преподавателей &
      Требуются внешние инструменты                          &
      Базовая статистика                                     &
      Ограничена оценками и дедлайнами                                                  \\
      \hline

      Помощь                                                 &
      IntelliSense и учебный материал                        &
      IntelliSense                                           &
      Общие подсказки                                        &
      Учебные материалы университета                                                    \\
      \hline

      Доступность для БГУИР                                  &
      Требуется разработать                                  &
      Свободное ПО                                           &
      Платно                                                 &
      Используется сейчас                                                               \\
      \hline
\end{longtable}}

Подобная система поднимет обучение на новый уровень, устранив любые траты
времени преподавателей и студентов на настройку локальных рабочих окружений на
своих устройствах, ускорит проверку и сдачу лабораторных работ и расширит
сценарии учебного взаимодействия преподавателей и студентов.

\subsection{Оценка аудитории}

Целевой аудиторией являются студенты и преподаватели университета. Она является
их обязательным средством коммуникации, а люди не относящиеся к университету,
доступа к ней не имеют. Сегменты аудитории описаны в таблице \ref{tab:audience},
а типичные их представители в таблице \ref{tab:personas}.

{\small\begin{longtable}{|>{\centering\arraybackslash}m{0.15\linewidth}|*{4}{>{\raggedright\arraybackslash}m{0.18\linewidth}|}}
      \caption{Целевая аудитория}\label{tab:audience}                             \\
      \hline
      \textbf{Критерий}                                                         &
      \textbf{Студенты-новички}                                                 &
      \textbf{Студенты-практики}                                                &
      \textbf{Преподаватели"-программисты}                                      &
      \textbf{Иные преподаватели}                                                 \\
      \hline
      \endfirsthead

      \caption*{Продолжение таблицы \ref{tab:audience}}                           \\
      \hline
      \textbf{Критерий}                                                         &
      \textbf{Студенты-новички}                                                 &
      \textbf{Студенты-практики}                                                &
      \textbf{Преподаватели"-программисты}                                      &
      \textbf{Иные преподаватели}                                                 \\
      \hline
      \endhead

      \hline
      \endfoot

      \hline
      \endlastfoot

      Характе"-ристика                                                            &
      Мало опыта, фокус на базовом программировании                             &
      Опытны, возможно, участвуют в коммерческих проектах, углублённое изучение &
      Преподаватели ответственные за лабораторные работы по написанию программ  &
      Преподаватели дисциплин не требующих написания кода                         \\
      \hline

      Технические навыки                                                        &
      Основы некоторого языка (C, C\# или Python), незнакомы с git              &
      Знакомы со множеством языков и технологий                                 &
      Профессиональные инженеры                                                 &
      Вне прикладного программирования                                            \\
      \hline

      Основная потребность                                                      &
      Пошаговые инструкции и автоматическая настройка сред                      &
      Интеграция учебных и рабочих проектов                                     &
      Автоматизация проверки кода                                               &
      Простота загрузки материалов                                                \\
      \hline

      Пример проблемы                                                           &
      \textquote{Не понимаю, как установить swift!}                             &
      \textquote{Мне нужно одновременно работать с  и Y.}                       &
      \textquote{Не хочу разбираться как запустить курсовую студента.}          &
      \textquote{Хочу отправить задание студентам.}                               \\
      \hline
\end{longtable}}

{\small\begin{longtable}{|>{\centering\arraybackslash}m{0.11\linewidth}|*{4}{>{\raggedright\arraybackslash}m{0.19\linewidth}|}}
      \caption{Профили персонажей}\label{tab:personas}                                                  \\
      \hline
      \textbf{Свойство}                                                                               &
      \textbf{Студенты-новички}                                                                       &
      \textbf{Студенты-практики}                                                                      &
      \textbf{Преподаватели"-программисты}                                                            &
      \textbf{Иные преподаватели}                                                                       \\
      \hline
      \endfirsthead

      \caption*{Продолжение таблицы \ref{tab:personas}}                                                 \\
      \hline
      \textbf{Свойство}                                                                               &
      \textbf{Студенты-новички}                                                                       &
      \textbf{Студенты-практики}                                                                      &
      \textbf{Преподаватели"-программисты}                                                            &
      \textbf{Иные преподаватели}                                                                       \\
      \hline
      \endhead

      \hline
      \endfoot

      \hline
      \endlastfoot

      Имя                                                                                             &
      Витька Лаптев                                                                                   &
      Мишаня Мишутков                                                                                 &
      Ольга Сергеевна                                                                                 &
      Анатолий Вадимович                                                                                \\
      \hline

      Демо"-графия и психо"-логия                                                                         &
      18 лет, ФИТУ, студент 1 курса, ничего по специальности не знает                                 &
      22 года, ФКП, студент 4 курса, работает, ни разу не притрагивался к университетским компьютерам &
      33 лет, КИ, ведёт лекции и лабораторные работы по C\#, долго проверяет работы                   &
      49 лет, КВМ, ведёт лекции по математическому анализу, любит внезапные тесты                       \\
      \hline

      Цель                                                                                            &
      Установить окружение для разработки на C на Windows                                             &
      Написать диплом                                                                                 &
      Просмотреть 30 курсовых работ                                                                   &
      Набрать лекционный материал                                                                       \\
      \hline

      Мотива"-ция                                                                                       &
      Надежда, что всё сразу заработает                                                               &
      Git/C++/Docker-интеграции                                                                       &
      Автоматическая проверка                                                                         &
      Общая среда написания документов                                                                  \\
      \hline

      Фрустра"-ция                                                                                      &
      \textquote{Не понимаю, куда делся после установки компилятор}                                   &
      \textquote{Хочу организованно держать все материалы по диплому в одном месте}                   &
      \textquote{Не собираюсь тратить на проверку работ все выходные}                                 &
      \textquote{Методичка для печати и статьи в СЭО нужно различно оформлять}                          \\
      \hline

      Взаимо"-действие                                                                                  &
      Домашний компьютер с Windows 10                                                                 &
      Личный ноутбук с Gentoo Linux                                                                   &
      Личный ноутбук и домашний компьютер с Windows 11                                                &
      Университетский компьютер с Windows 10 на кафедре и домашний с Windows 8.1                        \\
      \hline
\end{longtable}}

\end{document}
