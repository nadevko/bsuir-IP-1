\documentclass{bsuir}
\usepackage{makecell}
\usepackage{multirow}

\departmentlong{инженерной психологии и эргономики}
\worktitle{Лабораторная работа \textnumero4\\\textquote{Проектирование алгоритма работы пользователя}}
\titleleft{
    Проверила:\\
    Яцкевич А. Ю.\\
    ~
}
\titleright{
    Выполнил:\\
    Бородин А.Н.\\
    гр. 310901
}
\titlepageyear{2025}

\usepackage{pgfplots}
\usepackage{amsmath}
\usepackage{breqn}

\newlength{\tablewidth}
\setlength{\tablewidth}{\textwidth - \parindent}

\renewcommand*{\thesection}{\arabic{section}}

\begin{document}

\maketitle
\mainmatter

\textbf{Цель работы}: формирование умения составлять карту пользовательского пути (User flow) проектируемого приложения.

\textbf{Проектируемый продукт}: IDE, совмещенная с обучающей платформой для студентов БГУИР.

\section{Диаграммы}

Сценарий логина в систему представлен на рисунке \ref{img:1.png}. Диаграмма выполнения студентом лабораторной работы
изображена на рисунке \ref{img:2.png}.

\makeimage[User flow логина в систему]{1.png}

\makeimage[User flow выполнения лабораторной работы]{2.png}

\section{Обоснование информационной архитектуры}

Так как эта программа является расширением уже существующей, редактора Visual Studio Code, он заимствует его
архитектуру. В основе интерфейса этого редактора лежат задачи: открыть файл, скопировать текст, синхронизировать
репозиторий и т.д.

\end{document}