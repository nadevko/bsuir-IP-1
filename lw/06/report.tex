\documentclass{bsuir}

\departmentlong{инженерной психологии и эргономики}
\worktitle{Лабораторная работа \textnumero6\\\textquote{Презентация и защита проекта}}
\titleleft{
    Проверила:\\
    Яцкевич А. Ю.\\
    ~\\
    ~
}
\titleright{
    Выполнили:\\
    Бородин А.Н.\\
    Дриц М.Ф.\\
    гр. 310901
}
\titlepageyear{2025}

\renewcommand*{\thesection}{\arabic{section}}

\begin{document}

\maketitle

\mainmatter

\section{Введение}

Проект представляет собой разработку интегрированной среды разработки (IDE),
совмещенной с обучающей платформой, ориентированной на студентов и преподавателей
БГУИР. Целью проекта является создание единой экосистемы для обучения и разработки,
интегрированной с существующими ресурсами университета, для существенного повышения
эффективности учебного процесса. Программный продукт относится к сфере образовательных
технологий (EdTech) и инструментов для разработки программного обеспечения, охватывая
учебный процесс по дисциплилам, связанным с программированием, веб-разработкой,
встраиваемыми системами и микроконтроллерами.

\section{Актуальность и цель проекта}

\subsection{Актуальность: Анализ проблем}

Анализ текущей ситуации выявляет ряд проблем, с которыми сталкиваются участники
образовательного процесса. Ключевой проблемой является фрагментация инструментов:
необходимость постоянного переключения между различными IDE (Visual Studio, Eclipse,
PyCharm), Системой электронного обучения (СЭО), браузерами и эмуляторами приводит
к снижению эффективности и потере концентрации. Доступ к учебным материалам и
заданиям часто затруднен. Технические барьеры при настройке окружения для
специфических задач, таких как embedded-разработка, отнимают значительное время
и создают сложности. Преподаватели затрачивают существенные временные ресурсы
на рутинную проверку заданий, а анализ прогресса студентов и оперативная обратная
связь затруднены. Проект направлен на системное решение данных проблем, выявленных
через опросы студентов БГУИР, анализ их рабочих процессов и фидбэк от преподавателей.

\subsection{Цель проекта: Формирование единой эффективной экосистемы}

Цель проекта заключается в создании единого, интуитивно понятного и высокоэффективного
инструмента. Разрабатываемая экосистема объединяет процессы разработки и обучения,
интегрируя возможности IDE с образовательными ресурсами университета и автоматизируя
множество стандартных задач для создания единого пространства обучения и преподавания.

\section{Обзор функциональных возможностей и внутренняя логика}

В качестве основы решения выбрана среда разработки Visual Studio Code (VS Code) как
кроссплатформенный, открытый и расширяемый редактор. Функционал проекта расширяет
базовые возможности VS Code, добавляя компоненты, критически важные для образовательного
процесса. Основная логика реализована в виде интегрированной системы, состоящей
концептуально из трех взаимодействующих плагинов (расширений) для VS Code: плагина
для интеграции с сервисами университета, плагина для инструментов преподавателя
и плагина для управления контейнерными окружениями. Взаимодействие пользователя
с системой осуществляется преимущественно через единое боковое меню "Обучение БГУИР".

\subsection{Интеграция с СЭО БГУИР и внешними сервисами}

Ключевой функциональностью является глубокая интеграция с Системой электронного
обучения (СЭО) БГУИР на базе Moodle (lms.bsuir.by). Специализированный плагин для
интеграции с сервисами взаимодействует с API Moodle, обеспечивая автоматическую
синхронизацию заданий, лекций и дедлайнов непосредственно в интерфейсе редактора.
Плагин периодически или по запросу пользователя отправляет запросы к API Moodle для
получения/отправки данных. Это позволяет студентам получать актуальную информацию
по курсам без необходимости переключения в браузер. Преподавателям предоставляется
возможность создавать и публиковать учебные материалы и задания непосредственно из IDE.
Интеграция системы автоматической проверки работ на основе unit-тестов, запускаемых
в контейнерах, ускоряет процесс оценки и предоставления обратной связи.

Дополнительно реализована интеграция с GitHub для поддержки совместной разработки
и контроля версий, а также для хостинга проектов, обучая студентов работе с
индустриальными стандартами. Также реализована интеграция с системой IIS БГУИР
(iis.bsuir.by) для доступа к данным об оценках по предметам непосредственно в
рабочей среде.

\subsection{Управление рабочим окружением}

Для обеспечения стандартизированных и воспроизводимых рабочих окружений используется
технология контейнеризации, Docker. Для каждого типа задания или дисциплины
может быть создан свой образ Docker с необходимыми компиляторами, библиотеками и
инструментами. При выполнении задания или отправке на проверку плагин управления
контейнерными окружениями запускает соответствующий контейнер, где выполняется
код студента и автоматическая проверка. Это гарантирует единообразие окружения и
устраняет проблемы с локальной настройкой. Преподаватели, используя плагин
инструментов преподавателя, имеют возможность разрабатывать или модифицировать
эти Docker-окружения для конкретных заданий, включая написание автотестов,
которые будут запускаться в контейнере.

\subsection{Поддержка различных типов заданий и автоматизация}

Платформа поддерживает различные типы учебных заданий: лабораторные работы с отчетами,
задачи по программированию, проектированию, практические задания (учет оценок).
Предоставляются соответствующие инструменты и шаблоны. Для задач по программированию
при начале работы автоматически создается новый репозиторий на GitHub или происходит
переход к существующему, если он уже создан и не требует обновления/исправления.
Интеграция с Git упрощает работу с контролем версий, включая частичную автоматизацию
(коммиты при сохранении) и контекстные подсказки. Реализована специализированная
поддержка embedded-разработки со встроенными эмуляторами и шаблонами.

\subsection{Создание и редактирование контента}

Для написания описаний заданий, теоретических материалов и отчетов студентов
используется WYSIWYG-редактор с возможностью экспорта содержимого в PDF. Этот редактор
доступен как преподавателям (через плагин инструментов преподавателя) для создания
учебных материалов (описания заданий, теория), так и студентам для оформления
отчетов по лабораторным работам.

\subsection{Обратная связь и аналитика}

Система обеспечивает автоматическую проверку кода с запуском unit-тестов в
контейнерах после отправки работы студентом и предоставлением отчета об ошибках.
Доступ к оценкам за работы предоставляется непосредственно в IDE. Интеграция с
IIS БГУИР позволяет преподавателям просматривать общие оценки студентов по предметам.
Преподавательский интерфейс, предоставляемый плагином инструментов преподавателя,
включает возможности просмотра и редактирования сданных работ, добавления комментариев
и выставления оценок. Студенты могут просматривать свои работы, комментарии
преподавателя и полученные оценки. Коммуникация по вопросам заданий реализована
через систему комментариев. Дашборд успеваемости группы доступен преподавателям
как расширенный интерфейс просмотра оценок за курс (буфер, открывающийся при
двойном нажатии на курс в боковой панели).

\section{Пользовательский интерфейс}

Визуальное представление интегрированной платформы реализовано в рамках интерфейса
Visual Studio Code, используя его стандартные элементы и добавляя новые
специализированные панели и представления.

\subsection{Боковая панель БГУИР}

Основным элементом интерфейса является новая боковая панель в VS Code. На ней
отображается список курсов. При раскрытии курса показывается дерево пунктов:
разделы теории, заданий и их группировки. Также на панели видны актуальные
задания с дедлайнами, уведомления, раздел с оценками (из работ и IIS) и ссылки
для связи с преподавателями/комментариев.

\subsection{Представление материалов и заданий}

При выборе пункта (теория, задание, тестирование) на боковой панели открывается
специальный буфер или представление в основной области редактора. Для теоретических
материалов отображается текст теории. Для заданий — их описание. Для тестирования
— интерфейс с вопросами и вариантами ответов и возможностью перехода между ними.
Для задач по программированию — кнопка "Начать работу" для создания окружения
(Docker) и нового репозитория на GitHub. Если работа уже начата, кнопка
обеспечивает переход в существующее окружение и репозиторий.

\subsection{Интерфейс для преподавателей}

Преподавателям доступен специальный режим редактирования для создания/редактирования
курсов, заданий, теоретических материалов, тестов и Docker-окружений. Интерфейс для
просмотра и оценки сданных работ студентов также интегрирован. Дашборд
успеваемости группы открывается как буфер при двойном нажатии на курс в боковой
панели.

\subsection{Дополнительные элементы интерфейса}

Важные события отображаются стандартными уведомлениями VS Code. Панель Git
дополнена подсказками. Интерфейс настройки окружения максимально упрощен.
Для обеспечения безопасности на стационарных компьютерах университета по умолчанию
включена возможность автоматического выхода из аккаунта. Этот выход может
инициироваться по завершению выполнения задания (например, после успешной сдачи
или истечения срока) или по истечении определенного таймера бездействия
пользователя в целях предотвращения несанкционированного доступа к данным
учетной записи.

\section{Технологии и архитектура}

Решение основано на модификации Visual Studio Code, выбранного за надежность,
открытость и расширяемость. Архитектура включает систему из концептуально
трех плагинов VS Code: для интеграции с внешними сервисами (API Moodle и IIS),
для инструментов преподавателя и для управления контейнеризацией (Docker).
Эти плагины взаимодействуют между собой для обеспечения целостности и
функциональности системы. Использование контейнеризации (Docker) обеспечивает
стабильные рабочие окружения. Совместная работа поддерживается интегрированным
Git и хостингом на GitHub, коммуникация — системой комментариев.

\section{Безопасность и надежность}

Система предусматривает механизмы аутентификации и авторизации через интеграцию
с университетскими сервисами для контроля доступа студентов и преподавателей к
соответствующим данным и функциям. Данные об успеваемости и работы студентов
хранятся в соответствии с политиками безопасности университета (в Moodle, IIS,
GitHub). Использование контейнеров обеспечивает изоляцию рабочих окружений,
повышая безопасность выполнения кода. Дополнительно, для повышения безопасности
на общих компьютерах, реализована функция автоматического выхода из аккаунта
по завершению задания или таймеру бездействия.

\section{Выводы: Преимущества и перспективы}

Проект IDE, совмещенной с обучающей платформой для студентов БГУИР, представляет
собой комплексное решение для повышения эффективности учебного процесса
программирования. Решаются фундаментальные проблемы разрозненности инструментов
и сложностей освоения практических навыков. Анализ подтверждает актуальность
и востребованность решения. Преимущества включают упрощение рабочего процесса
для студентов и преподавателей, повышение качества обратной связи и освоение
индустриальных инструментов. Дальнейшая разработка и реализация функционала
позволят создать единую, удобную и эффективную экосистему обучения в университете.

\end{document}
