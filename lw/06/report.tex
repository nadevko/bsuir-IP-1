\documentclass{bsuir}

\departmentlong{инженерной психологии и эргономики}
\worktitle{
    Лабораторная работа \textnumero6\\
    \textquote{Презентация и защита проекта}
}
\titleleft{
    Проверила:\\
    Яцкевич А. Ю.\\
    ~\\
    ~
}
\titleright{
    Выполнили:\\
    Бородин А.Н.\\
    Дриц М.Ф.\\
    гр. 310901
}
\titlepageyear{2025}

\renewcommand*{\thesection}{\arabic{section}}

\begin{document}

\maketitle

\mainmatter

\section{Обоснование необходимости}

Образовательный процесс в технических вузах, включая БГУИР, требует от студентов
не только теоретических знаний, но и уверенного владения современными
технологиями программирования. На практике это сопровождается рядом проблем:
разрозненность инструментов (разные IDE, системы тестирования, платформы для
теории и заданий, системы контроля версий) требует постоянной настройки и
переключения, замедляя обучение и снижая мотивацию. Преподаватели также
сталкиваются со сложностями: ручная проверка заданий и сбор данных из
разрозненных систем отнимают время, мешая сосредоточиться на качестве
преподавания. Создание единой среды, объединяющей процессы обучения и
разработки, является необходимостью. Разрабатываемая интегрированная платформа
позволит студентам сосредоточиться на задачах, а преподавателям "--- на
содержании обучения.

\section{Общее описание продукта}

Наш программный продукт дополняет существующую Систему электронного обучения
(СЭО) университета, предоставляя интегрированную среду для работы с учебным
контентом и задачами. Основой выбран Visual Studio Code (VS Code) благодаря его
популярности, кроссплатформенности и расширяемости. Функционал проекта расширяет
базовые возможности VS Code, добавляя компоненты, критически важные для
образовательного процесса.

Система реализована в виде трех взаимодействующих плагинов (расширений) для VS Code:

\begin{enumerate}

	\item Плагин для интеграции с сервисами университета, использующий
	      расширение "MDLCode" для взаимодействия с СЭО (Moodle) и другими системами
	      (ИИС) для синхронизации данных и оценок.

	\item Плагин для инструментов преподавателя, предоставляющий функционал для
	      создания курсов и проверки работ.

	\item Плагин для управления контейнерными окружениями, обеспечивающий
	      стандартизированные рабочие среды с помощью Docker.

\end{enumerate}

\section{Функциональные возможности и их реализация}

Платформа предоставляет комплекс функций для оптимизации учебного процесса.

\subsection{Интеграция с СЭО БГУИР и внешними сервисами}

Глубокая интеграция с СЭО Moodle (lms.bsuir.by) осуществляется через плагин,
опирающийся на расширение "MDLCode". Это обеспечивает автоматическую
синхронизацию заданий, лекций и дедлайнов в интерфейсе редактора, позволяя
получать актуальную информацию без переключения в браузер. Преподаватели могут
создавать и публиковать материалы и задания непосредственно из IDE. Интеграция с
системой автоматической проверки работ (unit"=тесты в контейнерах) ускоряет
оценку и обратную связь. Дополнительно реализована интеграция с GitHub для
контроля версий и хостинга, а также с ИИС БГУИР (iis.bsuir.by) для доступа к
общим оценкам.

\subsection{Управление рабочим окружением}

Для стандартизации и воспроизводимости окружений используется контейнеризация
(Docker). Для каждого типа задания создается Docker"=образ с необходимыми
инструментами. Плагин управления окружениями запускает соответствующий контейнер
для выполнения кода и автопроверки, гарантируя единообразие. Преподаватели могут
разрабатывать и модифицировать эти окружения и автотесты.

\subsection{Поддержка различных типов заданий и автоматизация}

Платформа поддерживает лабораторные работы с отчетами, задачи по
программированию, проектированию, практические задания. Предоставляются
соответствующие инструменты и шаблоны. Для задач по программированию
автоматически создается или открывается репозиторий на GitHub. Интеграция с Git
упрощает контроль версий, включая частичную автоматизацию коммитов. Реализована
поддержка embedded"=разработки со встроенными эмуляторами.

\subsection{Обратная связь и аналитика}

Система автоматически проверяет код с помощью unit"=тестов в контейнерах,
предоставляя отчет об ошибках. Оценки за работы доступны в IDE. Интеграция с ИИС
позволяет преподавателям просматривать общие оценки. Преподавательский интерфейс
включает просмотр и редактирование сданных работ, добавление комментариев и
выставление оценок. Студенты видят свои работы и комментарии. Коммуникация
реализована через систему комментариев. Дашборд успеваемости группы доступен
преподавателям.

\section{Пользовательский интерфейс}

Визуальное представление платформы реализовано в интерфейсе Visual Studio Code,
используя его стандартные элементы и добавляя специализированные панели и
представления. Это позволяет работать в привычной среде, минимизируя
переключение между приложениями.

\subsection{Боковая панель БГУИР}

Основной элемент интерфейса "--- новая боковая панель в VS Code. На ней
отображается список курсов, дерево пунктов (теория, задания), актуальные задания
с дедлайнами, уведомления, оценки (из работ и ИИС) и ссылки для связи с
преподавателями и комментариев.

\subsection{Представление материалов и заданий}

При выборе пункта на боковой панели открывается буфер или представление в
основной области редактора: текст теории, описание заданий, интерфейс
тестирования с вопросами и вариантами ответов. Для задач по программированию
кнопка \textquote{Начать работу} создает и открывает Docker"=окружение и
репозиторий на GitHub. Для не"=программистских работ предусмотрен простой
интерфейс для загрузки файлов и просмотра комментариев. Создание тестов и
заданий преподавателями осуществляется через визуальный редактор.

\subsection{Интерфейс для преподавателей}

Преподавателям доступен режим редактирования для создания и редактирования курсов,
заданий, материалов, тестов и Docker"=окружений. Интерфейс для просмотра и
оценки сданных работ студентов также интегрирован. Дашборд успеваемости группы
открывается как буфер при двойном нажатии на курс.

\subsection{Дополнительные элементы интерфейса}

Важные события отображаются стандартными уведомлениями VS Code. Панель Git
дополнена подсказками. Интерфейс настройки окружения упрощен. Для безопасности
на стационарных компьютерах реализован автоматический выход из аккаунта (после
сдачи задания или при бездействии).

\section{Пользовательский сценарий}

Представим студента Филимона, который учится на втором курсе. Раньше настройка
инструментов занимала много времени. Теперь:

Филимон Вскодович открывает VS Code, входит в учетную запись БГУИР. На боковой
панели \textquote{Обучение БГУИР} видит курсы и задания с дедлайнами. Выбирает
лабораторную работу по программированию, читает описание. Нажимает
\textquote{Начать работу} "--- система автоматически создает репозиторий на
GitHub и запускает Docker"=окружение. Филимон сразу пишет код. IDE автоматически
коммитит изменения.

Завершив работу, Филимон отправляет ее на проверку из интерфейса. Система
запускает unit"=тесты, формирует отчет об ошибках. Преподаватель получает
уведомление, просматривает код и комментирует в IDE. Филимон видит комментарии и
оценку. Преподаватель просматривает дашборд успеваемости группы.

Для тестов по темам, Филимон выбирает тест на боковой панели. Открывается
интерфейс с вопросами и вариантами ответов. Он выбирает ответы и отправляет
тест.

Для не"=программистских заданий (рефераты), Филимон выбирает пункт на боковой
панели, видит описание и загружает файл через простой интерфейс. Комментарии и
оценка преподавателя отображается в системе.

\section{Выводы: Преимущества и перспективы}

Проект IDE, совмещенной с обучающей платформой, повышает эффективность учебного
процесса программирования в БГУИР, решая проблемы разрозненности инструментов.
Анализ подтверждает актуальность решения. Преимущества: упрощение рабочего
процесса, повышение качества обратной связи, освоение индустриальных
инструментов. Дальнейшая разработка включает интеграцию с другими системами
(расписание), добавление аналитики и расширение типов заданий для создания
полноценной экосистемы обучения.

\end{document}
